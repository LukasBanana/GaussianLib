\documentclass{article}
\title{Getting Started with GaussianLib}
\author{Lukas Hermanns}
\date{\today}

\usepackage{listings}
\usepackage{color}
\usepackage{pxfonts}
\usepackage{geometry}
\usepackage[T1]{fontenc}
\usepackage{xspace}

\geometry{
	a4paper,
	left=15mm,
	right=15mm,
	top=20mm,
	bottom=20mm
}

\begin{document}

\definecolor{brightBlueColor}{rgb}{0.5, 0.5, 1.0}
\definecolor{darkBlueColor}{rgb}{0.0, 0.0, 0.5}

\def\gausslib{\textsc{GaussianLib}\xspace}

\lstset{
	language = C++,
	basicstyle = \footnotesize\ttfamily,
	commentstyle = \itshape\color{brightBlueColor},
	keywordstyle = \bfseries\color{darkBlueColor},
	stringstyle = \color{red},
	frame = single,
	tabsize = 4,
	showstringspaces=false,
	numbers=none
}

\maketitle


%----------------------------------------------------------------------------------------
%	INTRRODUCTION
%----------------------------------------------------------------------------------------

\section*{Introduction}

The \gausslib is a simple C++ library for 2D and 3D applications.
It provides only basic linear algebra functionality for Vectors, Matrices, and Quaternions.


%----------------------------------------------------------------------------------------
%	COMPILATION
%----------------------------------------------------------------------------------------

\section*{Compilation}

In the following we consider to have a single C++ file named ``Example.cpp''.
More over \texttt{\%GaussianLibPath\%} denotes your GaussianLib installation directory.

\subsection*{GNU/C++}

The \gausslib requires \texttt{g++} version 4.8.1 or higher, with C++11 feature set enabled.
To compile your application with GNU/C++ (or MinGW on Windows), type this into a command line:
\begin{quote}
\texttt{g++ -I \%GaussianLibPath\%/include -std=c++11 Example.cpp -o ExampleOutput}
\end{quote}
If everything worked properly, your executable is named ``ExampleOutput''.

\subsection*{VisualC++}

The \gausslib requires VisualC++ 2013 (12.0) or higher, to support the C++11 features, which are used in the library.


%----------------------------------------------------------------------------------------
%	VECTORS
%----------------------------------------------------------------------------------------

\section*{Vectors}

There are three base classes for vectors: \texttt{Vector2T<T>}, \texttt{Vector3T<T>}, and \texttt{Vector4T<T>},
where \texttt{<T>} specifies the template typename T. There are also pre-defined type aliases
(\texttt{\textit{N}} is either 2, 3 or 4):
\begin{itemize}
	\item \texttt{Vector\textit{N}} Is a type alias to \texttt{Vector\textit{N}T<Real>}, where \texttt{Real} is either
		from type \texttt{float} or \texttt{double}.
	\item \texttt{Vector\textit{N}f} Is a type alias to \texttt{Vector\textit{N}T<float>}.
	\item \texttt{Vector\textit{N}d} Is a type alias to \texttt{Vector\textit{N}T<double>}.
	\item \texttt{Vector\textit{N}i} Is a type alias to \texttt{Vector\textit{N}T<int>}.
	\item \texttt{Vector\textit{N}ui} Is a type alias to \texttt{Vector\textit{N}T<unsigned int>}.
	\item \texttt{Vector\textit{N}b} Is a type alias to \texttt{Vector\textit{N}T<char>}.
	\item \texttt{Vector\textit{N}ub} Is a type alias to \texttt{Vector\textit{N}T<unsigned char>}.
\end{itemize}

\begin{lstlisting}
#include <Gauss/Gauss.h>
#include <iostream>

int main()
{
	Gs::Vector3 a(1, 2, 3), b(4, 5, 6);
	
	std::cout << "a = " << a << std::endl;
	std::cout << "b = " << b << std::endl;
	std::cout << "a * b = " << a*b << std::endl;
	std::cout << "a . b = " << Dot(a, b) << std::endl;
	std::cout << "a X b = " << Cross(a, b) << std::endl;
	std::cout << "|a| = " << a.Length() << std::endl;
	std::cout << "a / |a| = " << a.Normalize() << std::endl;
	
	return 0;
}
\end{lstlisting}
The vector classes have not been generlaized as much as the matrix class. This is due to support the public members
\texttt{x}, \texttt{y}, \texttt{z}, and \texttt{w}. I.e. you are not restricted to the bracket operator \texttt{[]}
to access vector components:
\begin{lstlisting}
a.x = 2;
a.z = 3;
a[0] += 2; // equivalent to a.x += 2;
\end{lstlisting}


%----------------------------------------------------------------------------------------
%	MATRICES
%----------------------------------------------------------------------------------------

\section*{Matrices}

There is only a single general-purpose class for matrices
(except SparseMatrix4T, see section \ref{sec:sparse_matrices}{Sparse Matrices}):
\texttt{Matrix<T, Rows, Cols>}, where \texttt{T} specifies the template typename T, \texttt{Rows} specifies the
number of rows of the matrix, and \texttt{Cols} specifies the number of columns of the matrix.

\begin{lstlisting}
#include <Gauss/Gauss.h>
#include <iostream>

int main()
{
	Gs::Matrix4 a(1, 2, 3), b(4, 5, 6);
	
	std::cout << "a = " << a << std::endl;
	std::cout << "b = " << b << std::endl;
	std::cout << "a * b = " << a*b << std::endl;
	std::cout << "a . b = " << Dot(a, b) << std::endl;
	std::cout << "a X b = " << Cross(a, b) << std::endl;
	std::cout << "|a| = " << a.Length() << std::endl;
	std::cout << "a / |a| = " << a.Normalize() << std::endl;
	
	return 0;
}
\end{lstlisting}


%----------------------------------------------------------------------------------------
%	SPARSE MATRICES
%----------------------------------------------------------------------------------------

\section*{Sparse Matrices}
\label{sec:sparse_matrices}

In 3D applications a 4x4 matrix is frequently used for transformations of 3D models.
However, with many 3D models, such transformations require a lot of memory.
Moreover, the 4th row of these 4x4 matrices is always $(0, 0, 0, 1)$ --- assumed that the transformation
only consists of translations, rotations, and scaling.

To reduce the memory footprint (and some computations) the \gausslib provides the \texttt{SparseMatrix4T<T>} class,
where the 4th row is implicit:
\begin{lstlisting}
#include <Gauss/Gauss.h>

int main()
{
	Gs::SparseMatrix4 m = Gs::SparseMatrix4::Identity();
	
	m.Translate(Gs::Vector3(0, 4, -2));
	m.RotateX(M_PI*0.5);
	m.RotateFree(Gs::Vector3(1, 1, 1), M_PI*1.5);
	m.Scale(Gs::Vector3(1, 0.5, 2));
	m.MakeInverse();
	
	Gs::Vector3 v(0, 0, 0);
	auto a = m.Transform(v); // Rotate and Translate (with implicit v.w = 1)
	auto b = m.Rotate(v); // Only rotate
	
	return 0;
}
\end{lstlisting}


%----------------------------------------------------------------------------------------
%	QUATERNIONS
%----------------------------------------------------------------------------------------

\section*{Quaternions}

Quaternions have the four components \texttt{x}, \texttt{y}, \texttt{z}, and \texttt{w} just like Vector4.
In contrast to vectors, quaterions can only have floating-point components.
\begin{lstlisting}
#include <Gauss/Gauss.h>

int main()
{
	Gs::Quaternion q0, q1; // Equivalent to Gs::QuaternionT<Gs::Real>
	Gs::Quaternionf qFloat;
	Gs::QuaternionT<double> qDouble;
	
	// Spherical Linear intERPolation (SLERP) between q0 and q1
	auto q2 = Slerp(q0, q1, 0.5);
	
	// Convert to 3x3 matrix
	Gs::Matrix3 rotation = q2.ToMatrix3();
	
	// Store rotation of quaterion in the left-upper 3x3 matrix of the sparse 4x4 matrix 'transform'
	Gs::SparseMatrix4 transform;
	Gs::QuaternionToMatrix(transform, q2);
	
	return 0;
}
\end{lstlisting}


%----------------------------------------------------------------------------------------
%	SWIZZLE OPERATOR
%----------------------------------------------------------------------------------------

\section*{Swizzle Operator}

For the three vector classes, there is support for the \textit{swizzle operator} (like in shading languages):
\begin{lstlisting}
// Enable `swizzle operator'
#define GS_ENABLE_SWIZZLE_OPERATOR

#include <Gauss/Gauss.h>

int main()
{
	Gs::Vector4 a, b;
	Gs::Vector3 c, d
	Gs::Vector2 e, f;
	
	e = a.xy();
	f = a.zz();
	c = a.xxz();
	b = a.xyxy();
	
	a.yz() += e;
	a.zx() *= 2;
	
	a = e.xxyy();
}
\end{lstlisting}
Every combination is possible!


%----------------------------------------------------------------------------------------
%	SHADING LANGUAGES
%----------------------------------------------------------------------------------------

\section*{Shading Languages}

There are two extra header files, which can be included optionally:
\begin{lstlisting}
#define GS_ENABLE_SWIZZLE_OPERATOR
#include <Gauss/Gauss.h>

// Includes all type aliases with name conventions of the DirectX High Level Shading Language HLSL.
#include <Gauss/HLSLTypes.h>

// Includes all type aliases with name conventions of the OpenGL Shading Language (GLSL).
#include <Gauss/GLSLTypes.h>

int main()
{
	// HLSL types
	float4x4 m0;
	double2x3 m1;
	int3 v0;
	
	// GLSL types
	mat4 m2 = m0;
	ivec2 v1 = v0.yz();
	ivec3 v2 = v0.xxy();
	
	return 0;
}
\end{lstlisting}


%----------------------------------------------------------------------------------------
%	FINE TUNING
%----------------------------------------------------------------------------------------

\section*{Fine Tuning}

By default, all vectors, quaternions, and matrices are initialized. To increase performance by not automatically
initialize this data, add the following to your compiler pre-defined macros:
\begin{quote}
\texttt{GS\_DISABLE\_AUTO\_INIT}
\end{quote}
If you don't want to disable the automatic initialization overall, you can explicitly construct a data type
who is uninitialized. This can be done with the \texttt{UninitializeTag} tag:
\begin{lstlisting}
Gs::Matrix4 m(Gs::UninitializeTag{});
\end{lstlisting}
\texttt{UninitializeTag} is an empty \texttt{struct}, so no memory will be allocated. It's just a hint to the compiler,
to call another constructor, which does no initialization.
Note, that uninitialized data should always be explicitly marked as such!




\end{document}